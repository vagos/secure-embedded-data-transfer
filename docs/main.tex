\documentclass[conference]{IEEEtran}
\IEEEoverridecommandlockouts
% The preceding line is only needed to identify funding in the first footnote. If that is unneeded, please comment it out.
\usepackage{cite}
\usepackage{amsmath,amssymb,amsfonts}
\usepackage{algorithmic}
\usepackage{graphicx}
\usepackage{textcomp}
\usepackage{xcolor}
\usepackage{hyperref}
\usepackage{cleveref}
\usepackage{dirtytalk}
\usepackage{listings}
\usepackage[outputdir=build]{minted}

\usemintedstyle{default}
\setminted{linenos}
\setminted{frame=single}
\setminted{framesep=2mm}

\usepackage[T1]{fontenc}
\renewcommand{\ttdefault}{cmtt}

\begin{document}

\title{Networked Embedded Systems Practical Project}

\author{\IEEEauthorblockN{Siyan Chen}
\IEEEauthorblockA{\textit{Technical University of Denmark} \\
siyan@chen.com}
\and
\IEEEauthorblockN{Evangelos Lamprou}
\IEEEauthorblockA{\textit{Technical University of Denmark} \\
e.lamprou@upnet.gr}
}

\maketitle

\begin{abstract}
    In this project, we present creation of an
    embedded network system using the popular ESP32 platform. The system serves
    the purpose of recording audio samples and facilitating their remote
    transmission over the Internet. We go into the design and implementation
    of the sytem while simultaneously putting attention
    on pivotal security considerations that touch multiple
    layers of the architecture. We offer an examination of
    the developmental journey, drawing attention to notable experiences and
    insights gleaned during deployment and make proposals for their improvement.
\end{abstract}

\begin{IEEEkeywords}
embedded network systems, security
\end{IEEEkeywords}

\section{Introduction}

This report is structured as follows: in \cref{sec:background}, 
we give a short overview of our project's goals while providing references to similar works. 
In \cref{sec:system_design}, we go into the design of the system, giving a high-level overview of the architecture and 
the components used. Then, in \cref{sec:system_implementation}, we go over the final implementation and comment on it's performance, 
while also highligthing some of the security considerations we took into account. Finally, in \cref{sec:discussion_and_conclusions},
we discuss the results of our work and propose some ideas for future work.

\section{Project Objectives} % Project Objectives 
\label{sec:background}

The goal of this project is to create a networked embedded system that can record audio samples and transmit them over the Internet.
The use case for something like could be a remote monitoring system, where the device is placed in a remote location and 
the user can listen to an audio feed from their device. Applications such as baby monitors or home security systems come to mind.
In both of these cases, the confidentiality of the audio feed is very important and
attacks on such systems have been demonstrated in the past \cite{BabyMonitorHack, VideoSurvAttacks}.

\section{System Design}
\label{sec:system_design}

\subsection{Architecture}
\subsection{Hardware Setup}
The system consists of two ESP32\cite{ESP32_Manual} microcontrollers.
\subsection{Software and Firmware}

For programming the ESP32 devices, we used the Espressif Systems' ESP-IDF\cite{ESP-IDF}
software development environment. The SDE is written in the C programming 
language. For cross-compiling our source code to the ESP platform we used the 
\texttt{xtensa-esp32-elf} compiler which is based on GCC.


\subsection{Communcation}

Communication between the two devices is done using the Message Queuing Telemetry Transport (MQTT) protocol\cite{MQTT_Survey}.
MQTT is a lightweight publish/subscribe messaging protocol that is designed for constrained devices and low-bandwidth.
For our project, we used a free public MQTT broker\footnote{hiveMQ: \url{https://www.hivemq.com/public-mqtt-broker/}}.

\section{System Implementation}
\label{sec:system_implementation}

\subsection{Security Analysis}
\label{subsec:security_analysis}


\subsubsection{Threat Model}

We assume a strong software adversary that has physical acess to the device.
A persistent attacker might be able to extract the firmware from the device and
analyze it. Such analysis can result in cryptographic keys and certificates being extracted from the device
\footnote{In \cref{appendix:reverse_engineering} we provide a tutorial-like section where we act as an attacker and attempt to reverse engineer the firmware of an ESP32 device.}, 
to the proprietary program logic of the device being reverse engineered.
In addition, the device might be placed in an untrusted environment, where the communication as well 
as other networking infrastructure might tampered with.
An adversary might be able to \say{listen} to the communication between the two devices,
as they might be inside the same network.
Thus, the adversary might be able to perform a wide range of network attacks such 
as a DNS Cache Poisoning attack \cite{Dissanayake_2018}, the adversary might have acess to the local router 
settings, making a Domain Hijacking and Redirection attack \cite{DnsHijacking} trivial. In both of these attacks, 
the data transferred from the recording device to the server might be redirected to a malicious server, 
where the data might be stored and analyzed.

\subsubsection{Security Considerations}

\paragraph{Communication}

In order to make communication between the two devices and the MQTT broker
trusted, we initiate an SSL (Secure Socket Layer) connection to the public broker. 
The use of SSL/TLS (Transport Layer Security)
ensures the confidentiality and integrity of data transmission. 
It achieves this by encrypting the data in transit, preventing
unauthorized access and eavesdropping. The SSL connection also provides a means
to verify the authenticity of the MQTT broker.
The use of certificates, and thus assymetric encryption might be inappropriate in the scenario where more 
memory contrained devices are part of the network. 
An alternative is the use of Transport Layer Security pre-shared key (TLS-PSK), where the two parties 
(device and broker) have a common key which they use to encrypt their communication channel.
However, this approach requires the establishment of a key exchange routine before communication can start.

\paragraph{Message Encryption}

Even if the communication between the two parties is (apparently) secured, data still needs to be encrypted with the possibility 
of the communication channel being compromised or the data being stored inside a server.
For message encryption, we use the Advanced Encryption Standard (AES) on cipher feedback (CFB) mode using 
a common 32-byte pre-shared key between the two devices.
The ESP32 offers hardware accelerated AES encryption and decryption routines\footnote{The ESP-IDF implementation of the AES routine can be found in \texttt{components/mbedtls/port/aes/block/esp\_aes.c}.}.
In our implementation, the key is stored inside the program's source code using 
a simple obsfucation function that shuffles the key's in a deterministic way before using it, essentially
using a different key that can not be \textit{easily} obtained from static inspection of the firmeware.
An alternative would be to use 
Each audio sample (a byte buffer of variable size) is encrypted using the key 
and subsequently sent.


\section{Discussion and Conclusions}
\label{sec:discussion_and_conclusions}

\section*{Acknowledgment}

\thanks{We'd like to thank the DTU PRG group for providing us with some of the hardware and equipment for this project.}

\bibliographystyle{IEEEtran}
\bibliography{IEEEabrv,bibliography}

\appendices
\crefalias{section}{appendix}

\appendix{Reverse Engineering ESP32 Firmware}
\label{appendix:reverse_engineering}

As proof of the necessity for taking precautions when shipping an embedded device, 
we provide a tutorial-like section where we act as an attacker and attempt to reverse engineer the firmware of an ESP32
device in order to extract sensitive information.

\begin{listing}[h]
\begin{minted}{Bash}
$ esptool.py read_flash $START $END flash.bin
$ ./esp32_image_parser.py show_partitions
\end{minted}
\end{listing}

\end{document}
